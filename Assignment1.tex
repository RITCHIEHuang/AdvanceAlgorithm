\documentclass[a4paper, 12pt, titlepage]{article}

\usepackage{amssymb}
\usepackage{amsmath}
\newtheorem{theorem}{Theorem}
\newtheorem{lemma}{Lemma}
\newtheorem{proof}{Proof}

\usepackage[noend]{algpseudocode}
\usepackage{algorithmicx,algorithm}
\renewcommand{\algorithmicrequire}{\textbf{Input:}}
\renewcommand{\algorithmicensure}{\textbf{Output:}}

\usepackage{mathtools}

\usepackage{graphicx}

\usepackage{indentfirst}
\setlength{\parindent}{2em}
\usepackage[CJKbookmarks]{hyperref}

\usepackage{enumerate}

\title{Problem Set 1 Solutions}
\author{ritchie huang}
\date{2019-09-25}

\begin{document}

\section{Problem 1}

\subsection{Modified Algorithm}

Given weighted graph ${\bf G(V, E), |V| = n}$. Assuming the min-cut is ${\bf C}$, and the weights of all edges in ${\bf C}$ is ${\bf W}$.
We define the weight of the edge ${\bf e_k}$ to ${\bf w_{k}}, k = 1, 2, \cdots, |E|$.The original contraction algorithm can be modified as below:
\begin{algorithm}[h]
    \caption{Modified Karger's Contraction Algorithm}
    \begin{algorithmic}[1]
        \Require {\bf G(V,E)}
        \Ensure {\bf Min-cut C}

        \Function{\bf Modified-Contraction}{G}
        % \State $\forall$ $w_{ij}$ in $G$, modify the edge with weight $w_{ij}$ to {\bf$w_{ij}$} parallel edges 
        %         between $v_{i}$ and $v_{j}$, the new graph ${\bf G'(V, E')}$ with $ W = \sum_{i < j} w_{ij}$ edges.
        \While{{\bf |V| > 2}}
            \State Select edge ${\bf e_{k}} \in {\bf E}$ with probability ${\bf \frac{w_{k}}{\sum_{j = 1}^{|E|} w_{j}}}, k = 1, 2, \cdots, |E|$;
            \State Merge the endpoints of ${\bf e_{k}}$, producing graph ${\bf G'}$;
            \State {\bf Modified-Contraction(${\bf G' }$)};
        \EndWhile

        \State \Return sets of edges remaining in the graph as ${\bf C}$
        \EndFunction
    \end{algorithmic}
\end{algorithm}


\subsection{Proof for the success probability of algorithm}

In the $i$-th ground contraction, the probability to choose an edge which is in the weighted min-cut {\bf C} is {\bf at most}:
\[
    \frac{W}{\frac{W (n - i)}{2}} = \frac{2}{n - i}
\]

The algorithm perform $n - 2$ contractions on the graph, we get this lower bound:
\begin{equation}
    \begin{aligned}
        Pr \big[ \text{return weighted min-cut \textbf{C} }\big] &\geq \prod_{i = 0}^{n - 2} (1 - \frac{2}{n - i}) \\
                                                 &= \prod_{i = 0}^{n - 3} \frac{ n - i - 2}{n - i} \\
                                                 &= \frac{2}{n ( n - 1)}
    \end{aligned}
\end{equation}


\section{Problem 2}

\subsection{Calculate the expected value }

% For any $S \in \mathcal{F} = \big\{ H \subseteq V : |H| = \frac{n}{2} \big\} $, they are chosen with same probability:
% \[
%     Pr[\text{Select S}] = \frac{1}{\binom{n}{\frac{n}{2}}}
% \]

% $\forall S \in \mathcal{F}, T = V \backslash S$, the expected cut size is:
% \begin{equation}
%     \begin{aligned}
%         E(S, T) &= \sum_{ e \in E} Pr \big[ (|e \cap S| = 1) \land (|e \cap T| = 1) \big] \\
%                 &= \sum_{ e \in E} \frac{1}{\binom{\frac{n}{2}}{1}} \cdot \frac{1}{\binom{\frac{n}{2}}{1}} \\
%                 &= \frac{4 m}{n^2}
%     \end{aligned}
% \end{equation}

For any $k \in \{  0, 1, 2, \cdots, m \}$, the probability of $|E(S, T)| = k$ should be:
\begin{equation}
    \begin{aligned}
        Pr \big[ |E(S, T)| = k \big] &= \frac{\binom{m}{k}}{2^{m}}
    \end{aligned}
\end{equation}


The expected value can be calculated as below:
\begin{equation}
    \begin{aligned}
        \mathbb{E} \bigg[| E(S, T) | \bigg] &= \sum_{k = 0}^{m} Pr \big[ | E(S, T) | = k] \cdot k \\
                               &= \sum_{k = 0}^{m} \frac{\binom{m}{k}}{2^{m}} \cdot k  \\
                               &= \frac{1}{2^{m}} \sum_{k = 0}^{m} \binom{m}{k} \cdot k \\
                               &= \frac{1}{2^{m}} 2^{m - 1} m \\
                               &= \frac{m}{2}
    \end{aligned}
\end{equation}

\subsection{Algorithm to generate subset}

\begin{algorithm}[h]
    \caption{Subroutine to generate Subset}
    \begin{algorithmic}[1]
        \Require $\mathcal{R(\cdot)}$, {\bf V}
        \Ensure S
        \Function{Generate}{$\mathcal{R(\cdot)}$, {\bf V}}
            \State Initialize {\bf S = $\emptyset$}
            \While{|{\bf S}| < $\frac{n}{2}$}
                \For{$v \in {\bf V}$}
                    \State $ p = \mathcal R(v)$
                    \If{$p \geq \frac{1}{2}$}
                        \State ${\bf S} = {\bf S} \cup \{ v \}$;
                    \EndIf
                \EndFor
            \EndWhile
            \State \Return {\bf S}
        \EndFunction
    \end{algorithmic}
\end{algorithm}


To analysis the correctness of the algorithm, we need to prove that:
$ \forall S \in \mathcal{F}$, the uniformal probability is given below:
\[
    \Pr [ \text{ Choose {\bf  S}} ] = \frac{1}{\binom{n}{\frac{n}{2}}}, \qquad \forall S \in \mathcal{F}.
\]

Given random variables generated by the algorithm, $\Pr [X_i = 1] = \frac{1}{2}, i = 1, 2, \cdots, n$, 
For any ${\bf S}$ is chosen only and if only when there exits $\frac{n}{2}$ random variables is equal to $1$.
Define $Y_i = \sum_{k = 0}^{n} X_{k}, i = 1, 2, \cdots, \binom{n}{\frac{n}{2}}$.
\begin{equation}
    \begin{aligned}
        \mathbb E \big[ Y_{i} = \frac{n}{2} \big] &= \mathbb E \big[ \sum_{k = 0}^{n} X_{k} = \frac{n}{2} \big] \\
                                                  &= \binom{n}{\frac{n}{2}} \mathbb E [X_i] \\
                                                  &= \frac{1}{2} \binom{n}{\frac{n}{2}} 
    \end{aligned}
\end{equation}
This is exactly the same with uniform distribution.

In the procedure of the algorithm, if we have generate a qualified subset $S$, the algorithm finish immediately, expected number of times
to call the $\mathcal{R(\cdot)}$ subroutine can be derived as below:
\begin{equation}
    \begin{aligned}
        \mathbb E \big[ N \big] &= \sum_{k = \frac{n}{2}}^{n} \Pr[ N_{k} = k] \cdot k \\
                                &= \sum_{k = \frac{n}{2}}^{n} \binom{k - 1}{\frac{n}{2} - 1} \big( \frac{1}{2} \big)^{k - 1} \cdot \frac{1}{2} \cdot k \\
                                &= \sum_{k = \frac{n}{2}}^{n} \binom{k - 1}{\frac{n}{2} - 1} \big( \frac{1}{2} \big)^{k} \cdot k \\
    \end{aligned}
\end{equation}

\section{Problem 3}
As vertices in the rooted tree has recursive structure.We consider to assign a polynomial to each vertex in the tree.
For earch vertex $v \in T$ with height $h$, the polynomial is defined as :
\[
    P_{v} = \begin{cases}
        x_{0}, \quad h = 0,  \\
        \prod_{i = 1}^{k} \left( x_{h} - P_{u_{i}} \right), h \geq 1 
    \end{cases}
\]

Given rooted tree $T$ with height $h$ and $n$ vertices, according to the polynomial defined above, there are $h + 1$ variables in $P_{root(T)} \rightarrow \left\{ x_{0}, x_{1}, \ldots, x_{h} \right\}$ , 
it's necessary for us to figure out the degree of $P_{root(T)}$.
Expand $P_{root(T)}$, we get:
\begin{equation}
    \begin{aligned}
        P_{root(T)} &= \left( x_{h} - P_{v_{1}} \right) \left( x_{h} - P_{v_{2}} \right) \cdots \left( x_{h} - P_{v_{k}}\right) \\
                    &= x_{h}^{k} + \ldots + \prod_{i = 1}^{k} P_{v_{i}}
    \end{aligned}
\end{equation}

$P_{root(T)}$ is then divided into 3 parts.We analysis the degree one by one.
\[
    \mathbf{Deg} \left( x_{h}^{k} \right) = k
\]
\[
    \mathbf{Deg} \left( \ldots \right) = \max \limits_{1 \leq i \leq k - 1} \sum_{k = 1}^{i} {\mathbf{Deg} \left( P_{v_{k}} \right)} + k - i
\]
\[
    \mathbf{Deg} \left( \prod_{i = 1}^{k} P_{v_{i}} \right) = \sum_{i = 1}^{k} \mathbf{Deg} \left( P_{v_{i}}\right) = n
\]
As $\forall i \in \{1, 2, \ldots, k \}, \mathbf{Deg} \left( P_{v_{i}} \right) \geq 1$. It's easy to prove that: 
\[
    n = \mathbf{Deg} \left( \prod_{i = 1}^{k} P_{v_{i}} \right) \geq \mathbf{Deg} \left( \ldots \right) \geq \mathbf{Deg} \left( x_{h}^{k} \right) = k
\]
where $n$ is the number of vertices in rooted tree $T$.

Now it's clear that $P_{root(T)}$ is a $h + 1$ multivariate polynomial with degree $n$.
We are going to construct polynomial for rooted trees $T_{1}$ and $T_{2}$, and prove that $T_{1}$ and $T_{2}$ are isomorphic is equivalent to $P_{root(T_{1})} \equiv P_{root(T_{2})}$.

Applying induction on the height of rooted trees.
\begin{enumerate}[1]
    \item If $h = 0$, it's trivial that $P_{root(T_{1})} = x_{0}$ and $P_{root(T_{2})} = x_{0}$, it's obvious $T_{1}$ and $T_{2}$ are isomorphic.
    \item Assume that $\forall h \leq k$, we have $P_{root(T_{2})} \equiv P_{root(T_{2})}$ equivalent to the isomorphism between $T_{1}$ and $T_{2}$.
    \item For $h = k + 1$, expand two polynomials, we have all terms are children vertices of $root(T_{1})$ and $root(T_{2})$, they are with heights $h = k$, can be verified by our hypothesis.
          Thus, it still works.
\end{enumerate}
Combining all of above, \textbf{we can test the isomorphism of rooted trees by test their polynomial identity}. 

We construct a polynomial as the subtraction of two polynomials:
\begin{equation}
    \begin{aligned}
        Q \left(x_{0}, x_{1}, \ldots, x_{h} \right) = P_{root(T_{1})} - P_{root(T_{2})}
    \end{aligned}
\end{equation}
We can test its identity on field $\mathbb{Z}_{p}$, where prime $p \geq 2n$.

According to \textbf{Schwartz-Zippel Theorem}, the error probability:
\[
    \Pr \left[ Q(r_{0}, r_{1}, \ldots, r_{h}) = 0 \right] \leq \frac{n}{p} \leq \frac{n}{2n} = \frac{1}{2}
\]        

We can independently perform such identity test for $\log{n}$ times, thus we can reduce the error probability to:
\[
    \left( \frac{1}{2} \right)^{\log{n}} = O \left( \frac{1}{n} \right)
\]

\section{Problem 4}
Consider we have a bijection $\sigma \in \mathbf{S_{n}}$, where $\mathbf{S_{n}}$ is the universe of permutations of set $\{1,2, \ldots, n\}$,
$\sigma$ maps $a_{i}$ to $b_{\sigma_{i}}$, $1 \leq i \leq n$.That's to say, if $a_{1}, \ldots, a_{n}$ is a permutation of 
$b_{1}, \ldots, b_{n}$, a $\sigma$ should be exist.
Therefore, we can define \textbf{Edmonds} Matrix \textbf{A}:

\[
    \mathbf{A}_{ij} = \begin{cases}
                x_{ij}, \qquad if\ a_{i} = b_{\sigma_{i}},\\
                0, \qquad otherwise.
             \end{cases}
\]

Matrix \textbf{A} has determinant: 
\[
    det(A) = \sum_{\sigma \in \mathbf{S_{n}}} sgn(\sigma) \prod_{i = 1}^{n} A_{i, \sigma_{i}}
\]
The determinant is a summation over $n!$ terms.It is helpful to analysis the result of $det(A)$, 
it is non-zero if and only if when $a_{1}, a_{2}, \ldots, a_{n}$ is a permutation of $b_{1}, b_{2}, \ldots, b_{n}$. 

Thus, We can construct a polynomial $\mathbf{Q}(x_{11}, x_{12}, \ldots, x_{nn}) = det(\mathbf{A})$.The problem of determination of
permutation now transform to determine if $\mathbf{Q}(x_{11}, x_{12}, \ldots, x_{nn})$ is identically zero or not.

To solve the problem, we first determine the degree of $n^2$ multivariate polynomial $\mathbf{Q}(x_{11}, x_{12}, \ldots, x_{nn}) \leq n$.
We can test its identity on the field $\mathbb{Z}_{p}$, where prime $p \geq 2n$.

According to \textbf{Schwartz-Zippel Theorem}, the error probability:
\[
    \Pr \left[ \mathbf{Q}(r_{11}, r_{12}, \ldots, r_{nn}) = 0 \right] \leq \frac{n}{p} \leq \frac{n}{2n} = \frac{1}{2}
\]             
The computation time complexity is dominated by the calculation of determinant, it's $O\left( n^{3} \right)$.

We can independently perform such identity test for $\log{n}$ times, thus we can reduce the error probability to:
\[
    \left( \frac{1}{2} \right)^{\log{n}} = O \left( \frac{1}{n} \right)
\]
Combining all of above together, we can solve this problem in $O \left( n^{3} \log{n} \right)$ with error probability $O \left( \frac{1}{n} \right)$.

\section{Problem 5}
The original problem can be transformed to the following form: 
\begin{equation}
    \begin{aligned}
        &\Pr \bigg[ e^{-\epsilon} Z \leq \hat{Z} \leq e^{\epsilon} Z \bigg] \geq 1 - \delta \\
        \Longleftrightarrow &\Pr \bigg[ \hat{Z} \leq e^{-\epsilon} Z \text{\ or \ }  \hat{Z} \geq e^{\epsilon} Z \bigg] \leq \delta \\
        \Longleftrightarrow &\Pr \bigg[ |\ln{\hat{Z}} - \ln{Z}| \geq \epsilon \bigg] \leq \delta \\
    \end{aligned}
\end{equation}

It's obvious that:

\begin{equation}
    \begin{aligned}
        \mathbb E \left[ \ln{\hat{Z}}\right] &= \mathbb E \left[ \sum_{i = 1}^{n} \hat{\rho_{i}} \right] \\
                                             &= \sum_{i = 1}^{n} \mathbb E \left[ \hat{\rho_{i}} \right] \\
                                             &= \sum_{i = 1}^{n} \mathbb E \left[ \frac{1}{s} \sum_{j = 1}^{s} X_{i}^{(j)} \right] \\
                                             &= \sum_{i = 1}^{n} \rho_{i} \\
                                             &= \ln{Z}
    \end{aligned}
\end{equation}

Applying Chebyshev's Inequality, we obtain:
\[
    \Pr \left[ |\ln{\hat{Z}} - \mathbb E \left[ ln{\hat{Z}} \right] | \geq \epsilon \right] \leq \frac{\mathbf{Var} \left[ \ln{\hat{Z}}\right] }{\epsilon^{2}}
\]

Therefore, we need to calculate the variance of $\ln{\hat{Z}}$.
\begin{equation}
    \begin{aligned}
        \mathbf{Var} \left[ \ln{\hat{Z}} \right] &= \mathbb E \left[ \left( \ln{\hat{Z}} \right)^2 \right] - \mathbb E^2 \left[ \ln{\hat{Z}} \right]\\
                                                 &= \mathbb E \left[ \left( \sum_{i = 1}^{n} \hat{\rho_{i}} \right)^2 \right] - \left( \sum_{i = 1}^{n} \rho_{i} \right)^2 \\
    \end{aligned}
\end{equation}

The first term can be expanded as below:
\begin{equation}
    \begin{aligned}
        \mathbb E \left[ \left( \sum_{i = 1}^{n} \hat{\rho_{i}} \right)^2 \right] &= \mathbb E \left[ \sum_{i = 1}^{n} \hat{\rho_{i}}^2 + 2 \sum_{1 \leq i < j \leq n} \hat{\rho_{i}} \cdot \hat{\rho_{j}}\right] \\
                                                                                  &= \sum_{i = 1}^{n} \mathbb E \left[ \hat{\rho_{i}}^2 \right] + 2 \sum_{1 \leq i < j \leq n} \mathbb E \left[ \hat{\rho_{i}} \cdot \hat{\rho_{j}} \right] \\
                                                                                  &= \sum_{i = 1}^{n} \frac{1}{s} \mathbb E^2 \left[ X_{i} \right] + 2 \sum_{1 \leq i < j \leq n} \mathbb E \left[ \hat{\rho_{i}} \right] \cdot \mathbb E\left[ \hat{\rho_{j}} \right] \\
                                                                                  &= \frac{1}{s} \sum_{i = 1}^{n}  \rho_{i}^2 + 2 \sum_{1 \leq i < j \leq n} \rho_{i} \cdot \rho_{j}
    \end{aligned}
\end{equation}

Combining all of above:
\begin{equation}
    \begin{aligned}
        \Pr \left[ |\ln{\hat{Z}} - \mathbb E \left[ ln{\hat{Z}} \right] | \geq \epsilon \right] &\leq \frac{\mathbf{Var} \left[ \ln{\hat{Z}}\right] }{\epsilon^{2}} \\
                                                                                                &= \frac{\frac{1}{s} \sum_{i = 1}^{n} \rho_{i}^2 + 2 \sum_{1 \leq i < j \leq n} \rho_{i} \cdot \rho_{j} - \left( \sum_{i = 1}^{n} \rho_{i} \right)^2 }{\epsilon^2} \\
                                                                                                &= \frac{\frac{1}{s} \sum_{i = 1}^{n} \rho_{i}^2 - \sum_{i = 1}^{n} \rho_{i}^2}{\epsilon^2} \\
                                                                                                &= \frac{\left( \frac{1}{s} - 1 \right) \sum_{i = 1}^{n} \rho_{i}^2}{\epsilon^2} \\
                                                                                                &= \delta
    \end{aligned}
\end{equation}
Solve the equation above, we get $s$:
\[
    s = \frac{1}{1 + \frac{\delta \epsilon^2}{\sum_{i = 1}^{n} \rho_{i}^2}} \\
      \geq \frac{1}{1 + \frac{4 \delta \epsilon^2}{n}}
\]

\section{Problem 6}

\end{document}