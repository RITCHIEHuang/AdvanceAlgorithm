\documentclass[a4paper, 12pt, titlepage]{article}
\usepackage{ctex} 
\usepackage{amssymb}
\usepackage{amsmath}
\newtheorem{theorem}{Theorem}
\newtheorem{lemma}{Lemma}
\newtheorem{proof}{Proof}

\usepackage[noend]{algpseudocode}
\usepackage{algorithmicx,algorithm}
\renewcommand{\algorithmicrequire}{\textbf{Input:}}
\renewcommand{\algorithmicensure}{\textbf{Output:}}

\usepackage{mathtools}

\usepackage{graphicx}

\usepackage{indentfirst}
\setlength{\parindent}{2em}
\usepackage[CJKbookmarks]{hyperref}

\usepackage{enumerate}

\title{Problem Set 2 Solutions}
\author{姓名: 黄睿 \\ 学号: MP1933008}
\date{2019-10-23}

\begin{document}

\maketitle

\section{Problem 1}

\subsection{Q1}
$\Pr \left[ X \geq t \right] \Leftrightarrow \Pr \left[ e^{\lambda X} \geq e^{\lambda t} \right], \text{for} \lambda \geq 0.$
Thus, the question tranformed to the equation on the right hand side.
\begin{equation}
    \begin{aligned}
        \Pr \left[ X \geq t \right] = \Pr \left[ e^{\lambda X} \geq e^{\lambda t} \right] &\leq \frac{\mathbb{ E } \left[ e^{\lambda X} \right] }{e^{\lambda t}} \quad \left( \text{Markov Inequality} \right) \\
                                                            &= \exp \left(- \left(\lambda t - \ln{\mathbb{E} \left[ e^{\lambda X} \right]} \right) \right) \\
                                                            &= \exp \left(- \left(\lambda t - \Psi_{X}(\lambda) \right) \right)                                                         
    \end{aligned}
\end{equation}

We have $\exp \left(- \left(\lambda t - \Psi_{X}(\lambda) \right) \right) \geq \exp \left( - \Psi_{X}^{*}(\lambda) \right)$, thus:
\[
    \Pr \left[ X \geq t \right] = \exp \left( - \Psi_{X}^{*}(\lambda) \right)
\]
For $F(\lambda) = \lambda t - \Psi_{X}(\lambda), \lambda \geq 0 $.If $\Psi_{X}(\lambda)$ is continuously differentiable, we can perform standard analysis of this function, taking gradient of both sides:
\[
    \nabla_{\lambda} F(\lambda) = t - \nabla_{\lambda} \Psi_{X}(\lambda)  
\]
Let the gradient equal to 0, we can find that the unique $\lambda \geq 0$ satisfying $\Psi_{X}'(\lambda) = t$, according to the convexity of $\Psi_{X}(\lambda)$, we obtain that: 
\[
    \Psi_{X}^{*}(t) = F(\lambda)|_{\Psi_{X}(\lambda) = t} = \sup_{\lambda \geq 0} \left( \lambda t - \Psi_{X}(\lambda) \right)  
\]

\subsection{Q2}

Gaussian random variable $\mathbf{X}$, it's probability density function is given by $f(x) = \frac{1}{\sqrt{2 \pi \sigma^2}} e^{-\frac{ (x - \mu )^2 }{2 \sigma^2}}$, so we have:

\begin{equation}
    \begin{aligned}
        \Psi_{X}(\lambda) &= \ln \mathbb E \left[ e^{\lambda X} \right] \\
                          &= \ln \int_{- \infty}^{\infty} e^{\lambda x} \frac{1}{\sqrt{2 \pi \sigma^2}} e^{-\frac{ (x - \mu )^2 }{2 \sigma^2}} \, dx \\
                          &= \ln \left[ \frac{1}{\sqrt{2 \pi \sigma^2}} \int_{- \infty}^{\infty} e^{\lambda x - \frac{(x - \mu)^2}{2 \sigma^2}} \,dx \right] \\
                          &= \ln \left[ \frac{1}{\sqrt{2 \pi \sigma^2}} \int_{- \infty}^{\infty} e^{- \frac{\left( x - (\lambda \sigma^2 + \mu) \right)^2 - \left( \lambda^2 \sigma^4 + 2 \lambda \mu \sigma^2 \right)}{2 \sigma^2}} \, dx \right] \\
                          &= \ln \left[ e^{\lambda \mu + \frac{\lambda^2 \sigma^2}{2}} \frac{1}{ \sqrt{2 \pi \sigma^2}} \int_{- \infty}^{\infty} e^{- \left( \frac{x - (\lambda \sigma^2 + \mu)}{ \sqrt{2} \sigma } \right)^2 } \, dx \right] \\
                          &= \ln \left[ e^{\lambda \mu + \frac{\lambda^2 \sigma^2}{2}} \frac{1}{ \sqrt{2 \pi \sigma^2}} \int_{- \infty}^{\infty} e^{- t^2 } \, d \left( \sqrt{2} \sigma t \right) \right] \quad \left(\text{let } t = \frac{x - (\lambda \sigma^2 + \mu)}{ \sqrt{2} \sigma } \right)\\
                          &= \ln \left[ e^{\lambda \mu + \frac{\lambda^2 \sigma^2}{2}} \frac{1}{\sqrt{\pi}} \int_{- \infty}^{\infty} e^{- t^2 } \, dt \right] \\
                          &= \ln \left[ e^{\lambda \mu + \frac{\lambda^2 \sigma^2}{2}} \right] \\
                          &= \lambda \mu + \frac{\lambda^2 \sigma^2}{2}
    \end{aligned}
\end{equation}

Thus, we can calculate $\Psi_{X}^{*}(t)$
\begin{equation}
    \begin{aligned}
        \Psi_{X}^{*}(t) &= \sup_{\lambda \geq 0} \left( \lambda t - \Psi_{X}(\lambda) \right) \\
                        &= \sup_{\lambda \geq 0} \left( \lambda t - \lambda \mu - \frac{\lambda^2 \sigma^2}{2} \right) \\
                        &= \frac{ \lambda^2 \sigma^2 }{2}
    \end{aligned}
\end{equation}

Now, the upper tail can be bounded:
\begin{equation}
    \begin{aligned}
        \Pr \left[ X \geq t \right] \leq \exp \left( - \frac{\lambda^2 \sigma^2}{2} \right)
    \end{aligned}
\end{equation}

\subsection{Q3}
Poisson random variable $\mathbf{X ~}$, it's probability distribution is given by $\Pr \left[ X = k \right] = e^{-\nu} \frac{\nu^{k}}{k!}$
We have:

\begin{equation}
    \begin{aligned}
        \Psi_{X}(\lambda) &= \ln \mathbb E \left[ e^{\lambda X} \right] \\
                          &= \ln \sum_{k = 0}^{\infty} \Pr \left[ X = k \right] e^{\lambda k} \\
                          &= \ln \sum_{k = 0}^{\infty} e^{ \lambda k - \nu } \frac{ \nu^{k}}{ k! } \\
                          &= \ln \left[ e^{ - \nu } \sum_{k = 0}^{\infty} e^{ \lambda k } \frac{ \nu^{k} }{ k! } \right] \\
                          &= \ln \left[ e^{ - \nu } \sum_{k = 0}^{\infty} \frac{\left( e^{ \lambda } \nu \right)^{k}}{ k! } \right] \\
                          &= \ln \left[ e^{ - \nu } e^{e^{\lambda} \nu } \right] \\
                          &= \left( e^{ \lambda } - 1 \right) \nu 
    \end{aligned}
\end{equation}

Then, we get $\Psi_{X}{*}(t)$:
\begin{equation}
    \begin{aligned}
        \Psi_{X}{*}(t) &= \sup_{\lambda \geq 0} \left( \lambda t - \Psi_{X}(\lambda) \right) \\
                       &= \sup_{\lambda \geq 0} \left( \lambda t - (e^{\lambda} - 1) \nu ) \right) \\
                       &= \lambda e^{\lambda} \nu - (e^{\lambda} - 1) \nu \\
                       &= \left( (\lambda - 1) e^{\lambda} + 1 \right) \nu 
    \end{aligned}
\end{equation}

According to $Q_1$, we have:
\begin{equation}
    \begin{aligned}
        \Pr \left[ X \geq t \right] \leq \exp \left( - \left( (\lambda - 1) e^{\lambda} + 1 \right) \nu  \right)
    \end{aligned}
\end{equation}


\subsection{Q4}
Bernoulli ranndom variable $\mathbf{X}$, it's probability distribution is given by $\Pr \left[ X = 1 \right] = 1 - \Pr \left[ X = 0 \right] = p $, thus we have:
\begin{equation}
    \begin{aligned}
        \Psi_{X}(\lambda) &= \ln \mathbb E \left[ e^{\lambda X} \right] \\
                          &= \ln \left[ p e^{\lambda} + (1 - p) \right] \\
    \end{aligned}
\end{equation}

Then, we get:
\begin{equation} \label{equation-9}
    \begin{aligned}
        \Psi_{X}^{*}(t) &= \sup_{\lambda \geq 0} \left( \lambda t - \Psi_{X}(\lambda) \right) \\
                       &= \sup_{\lambda \geq 0} \left( \lambda t - \ln \left[ p e^{\lambda} + (1 - p) \right] \right) \\
    \end{aligned}
\end{equation}
For the equation above, taking derivative w.r.t $\lambda$, we have:
\[
    t = \frac{ e^{\lambda} p }{ e^{\lambda} + 1 - p }
\]

We may solve $\lambda$:
\[
    \lambda = \ln \left[ \frac{ (1 - p) t }{ (1 - t) p } \right]
\]

Thus, we may combing with equation \ref{equation-9}:
\begin{equation}
    \begin{aligned}
        \Psi_{X}^{*}(t) &= \ln \left[ \frac{ (1 - p) t }{ (1 - t) p } \right] t - \ln \left[ p \frac{ (1 - p) t }{ (1 - t) p } + (1 - p) \right] \\
                        &= (1 - t) \ln \frac{ 1 - t }{1 - p} + t \ln \frac{t}{p}
    \end{aligned}
\end{equation}


\subsection{Q5}
As $X_1, X_2, \cdots, X_n$ are i.i.d random variables, we have:
\begin{equation}
    \begin{aligned}
        \Psi_{X}(\lambda) &= \ln \mathbb E \left[ e^{\lambda X} \right] \\
                           &= \ln \mathbb E \left[ e^{\lambda \sum_{i = 1}^{n} X_{i}} \right] \\
                           &= \ln \prod_{i = 1}^{n} \mathbb E \left[ e^{\lambda X_{i} } \right] \\
                           &= \sum_{i = 1}^{n} \ln \mathbb E \left[ e^{\lambda X_{i} } \right] \\
                           &= \sum_{i = 1}^{n} \Psi_{X_{i}}(\lambda)
    \end{aligned}
\end{equation}

Similarly, for $\Psi_{X}^{*}(t)$, we have:
\begin{equation}
    \begin{aligned}
        \Psi_{X}^{*}(t) &= \sup_{\lambda \geq 0} \left( \lambda t - \Psi_{X}(\lambda) \right) \\
                        &= \sup_{\lambda \geq 0} \left( \lambda t - \sum_{i = 1}^{n} \Psi_{X_{i}}(\lambda) \right) \\
                        &= \sum_{i = 1}^{n} \sup_{\lambda \geq 0} \left( \lambda \frac{t}{n} - \Psi_{X_{i}}(\lambda) \right) \\
                        &= \sum_{i = 1}^{n} \Psi_{X_{i}}^{*}(\frac{t}{n}) \\
                        &= n \Psi_{X_{i}}^{*}(\frac{t}{n}) \quad (i.i.d)
    \end{aligned}
\end{equation}

For Binomial random variable $X \sim Bin(n, p)$, it can be decomposed as sum of $ n \text{\, i.i.d random Bernoulli random variables\, } X_{1}, X_{2}, \cdots, X_{n} $.
According to what we have above, the upper bound can be measured:
\begin{equation}
    \begin{aligned}
        \Pr \left[ X \geq t \right] &\leq \exp \left( - \Psi_{X}^{*}(t) \right) \\
                                    &= \exp \left( - n \Psi_{X_{i}}^{*}(\frac{t}{n}) \right) \\
                                    &= \exp \left( - n D(Y || X_{i}) \right)  
    \end{aligned}
\end{equation}
Where $Y \in \{ 0, 1 \}$ is a Bernoulli random variable with parameter $ \frac{t}{n} $.

Given geometric random variable $\mathbf{X}$ with distribution $\Pr \left[ X = k \right] = ( 1- p)^{k -1} p$.
\begin{equation}
    \begin{aligned}
        \Psi_{X}(\lambda) &= \ln \mathbb E \left[ e^{\lambda X} \right] \\
                          &= \ln \sum_{k = 1}^{\infty} \Pr \left[ X = k \right] e^{\lambda k} \\
                          &= \ln \sum_{k = 1}^{\infty} e^{\lambda k} (1-p)^{k - 1} p \\
                          &= \ln \frac{e^{\lambda } p}{e^{\lambda } (p-1)+1} \\
    \end{aligned}
\end{equation}

\begin{equation}
    \begin{aligned}
        \Psi_{X}^{*}(t) &= \sup_{\lambda \geq 0} \left( \lambda t - \Psi_{X}(\lambda) \right) \\
                        &= \sup_{\lambda \geq 0} \left( \lambda t - \ln \frac{e^{\lambda } p}{e^{\lambda } (p-1)+1} \right) \\
                        &= t \ln \left(\frac{1-t}{(p-1) t}\right)-\ln \left(-\frac{p (t-1)}{p-1}\right)
    \end{aligned}
\end{equation}

Combining all of them together:
\begin{equation}
    \begin{aligned}
        \Pr \left[ X \geq t \right] &\leq \exp \left( - \Psi_{X}^{*}(t) \right) \\
                                    &= \exp \left( - n \Psi_{X_{i}}^{*}(\frac{t}{n}) \right) \\
                                    &= \exp \left( - t \ln \left(\frac{n-t}{(p-1) t}\right) + n \ln \left(-\frac{p (t-n)}{n(p-1)}\right) \right)  
    \end{aligned}
\end{equation}

\section{Problem 2}

We define any vertex $u \in V(Q_n)$ as n independent random variables $(X_{1}, X_{2}, \cdots, X_{n}) $ where $X_{i} \in \{ 0, 1 \}, i = 1, 2, \cdots, n $.

Next, we can prove the function $f(X_{1}, X_{2}, \cdots, X_{n})$ satisfying the Lipschitz condition. For any $X_{1}, X_{2}, \cdots, X_{n}$ and any $Y_{i} \in \{ 0, 1 \}, i = 1, 2, \cdots, n$.
We denote vertex $\mathbf{u}$ as $(X_{1}, X_{2}, \cdots, X_{i - 1}, X_{i}, \cdots, X_{n})$, and vertex $\mathbf{w}$ as \\$(X_{1}, X_{2}, \cdots, X_{i - 1}, Y_{i}, \cdots, X_{n})$.According to
the definition of shortest distance, we know that there's an edge between vertex $\mathbf{u}$ and vertex $\mathbf{w}$.\\ Thus, we have:
\[
    1 + f(\mathbf{u}) \leq f(\mathbf{w}) 
\]
Symmetrically, we have:
\[
    1 + f(\mathbf{w}) \leq f(\mathbf{u})  
\]
Combining them, we have:
\begin{equation}
    \begin{aligned}
        | f(\mathbf{u}) - f(\mathbf{w}) | \leq 1
    \end{aligned}
\end{equation}

Now that $f$ satisfying Lipschitz condition with constants $1$, we can apply \textbf{Method of bounded differences}:
\begin{equation}
    \begin{aligned}
        \Pr \left[| f(\mathbf{X}) - \mathbb E \left[ f(\mathbf{X}) \right] | \geq t \sqrt{n \log{n}} \right] &\leq 2 \exp \left( - \frac{t^2 n \log{n}}{2 \sum_{i = 1}^{n} 1^2}\right) \\
                                                                                                             &= 2 \exp \left( - \frac{t^2 \log{n}}{2} \right) \\
                                                                                                             &= 2 n^{- \frac{t^2}{2} } \\
                                                                                                             &= n^{ \log_{n}{2} - \frac{t^2}{2}}           
    \end{aligned}
\end{equation}

Let $ \log_{n}{2} - \frac{t^2}{2} = -c$, we have: $c = \frac{t^2}{2} - \log_{n}{2}$.

\section{Problem 3}

\subsection{Q1}
It's obvious that $\forall 1 \leq i \leq n, \Pr \left[ Y_{i} = 1 \right] \leq p $, we can let $\Pr \left[ Y_{i} = 1 \right] = t_{i}, \quad \text{where} \quad t_{i} \leq p, i = 1, 2, \cdots, n$.Thus, $\forall 1 \leq i \leq n$, we generate $A$ as a uniform random variable between $\left[ 0, 1 \right]$, we can construct the
coupling $\mathcal{C}$ as below:
\[
    Y_{i} = \begin{cases}
        1, if \ A \leq t_{i} \\
        0, otherwise
    \end{cases}  
\]
\[
    X_{i} = \begin{cases}
        1, if \ A \leq p \\
        0, otherwise
    \end{cases}  
\]
We can easily verify that $\forall 1 \leq i \leq n, Y_{i} \leq X_{i}$, formally:
\[
    \Pr_{\mathcal{C}} \left[ \forall 1 \leq i \leq n, Y_{i} \leq X_{i} \right] = 1
\]
According to stochastic dominance, we have:
\[
    \forall a > 0, \Pr \left[ \sum_{i = 1}^{n} Y_{i} \geq a \right] \leq \Pr \left[ \sum_{i = 1}^{n} X_{i} \leq a \right] 
\]

\subsection{Q2}
Now that $X_{1}, X_{2}, \cdots, X_{n}$ are mutually independent, by linearity of expectation, it holds that $\mathbb E \left[ \sum_{i = 1}^{n} X_{i} \right] = np$.
\begin{equation}
    \begin{aligned}
        \Pr \left[ \sum_{i = 1}^{n} Y_{i} \geq np + t \right] &\leq  \Pr \left[ \sum_{i = 1}^{n} X_{i} \geq np + t \right] \\
                                                              &= \Pr \left[ \sum_{i = 1}^{n} X_{i} - \mathbb E \left[ \sum_{i = 1}^{n} X_{i} \right] \geq t \right] \\
                                                              &\leq \exp \left( - \frac{2 t^2}{n} \right) \quad (\text{Chernoff Bound})
    \end{aligned}
\end{equation}

\section{Problem 4}
\subsection{Q1}
The origin triangle inequality can be rewrited:
\begin{equation}
    \begin{aligned}
        &\quad \, d(A, B) + d(B, C) \geq d(A, C) \\
        &\Leftrightarrow 1 - sim(A, B) + 1 - sim(B, C) \geq 1 - sim(A, C) \\
        &\Leftrightarrow sim(A, B) + sim(B, C) - sim(A, C) \leq 1 \\
        &\Leftrightarrow \Pr_{h \in \mathcal F} \left[ h(A) = h(B) \right] + \Pr_{h \in \mathcal F} \left[ h(B) = h(C) \right] - \Pr_{h \in \mathcal F} \left[ h(A) = h(C) \right] \leq 1 \\
        &\Leftrightarrow \Pr_{h \in \mathcal F} \left[ h(A) = h(B) \right] + \Pr_{h \in \mathcal F} \left[ h(B) = h(C) \right] - \Pr_{h \in \mathcal F} \left[ (h(A) = h(B)) \wedge (h(B) = h(C)) \right] \leq 1 \\
        &\Leftrightarrow \Pr_{h \in \mathcal F} \left[ (h(A) = h(B)) \vee (h(B) = h(C)) \right] \leq 1
    \end{aligned}
\end{equation}
It's obvious that probability value is less or equal than $1$.

\subsection{Q2}
Assume that we have locality sensitive hash function family corresponding to Dice's coefficient, by triangle inequality we have:
\begin{equation}
    \begin{aligned}
        &\forall A, B, C \in 2^{U}, d(A, B) + d(B, C) \geq d(A, C) \\
        &\Leftrightarrow \frac{|A| + |B| - 2 |A \cap B|}{|A| + |B|} + \frac{|B| + |C| - 2 |B \cap C|}{|B| + |C|} \geq \frac{|A| + |C| - 2 |A \cap C|}{|A| + |C|} \\ 
    \end{aligned}
\end{equation}
Let $|A| = |C| = \frac{|U|}{2}, A = U - C, B = U$, we get:
\[
    \frac{\frac{|U|}{2}}{\frac{3|U|}{2}} + \frac{\frac{|U|}{2}}{\frac{3|U|}{2}} \geq 1 \Leftrightarrow \frac{2}{3} \geq 1 \quad (Contradiction)
\]
Thus, we can prove that nno locality sensitive hash function family corresponding to Dice's coefficient.
\subsection{Q3}
\end{document}